Qui di seguito vengono riportate le MACRO e le intestazioni dei metodi corrispondenti alle soluzioni implementate per il filtro FIR in questione. In particolare, ogni definizione presenta la documentazione associata.
\lstinputlisting[language=C]{definitions/definitions.h}
Bisogna, inoltre, specificare che ogni solution (tranne per alcune come verrà appositamente specificato) è stata progettata inizialmente in Xilinx\textsuperscript{\textregistered} Vivado\textsuperscript{\textregistered} High-Level Synthesis (HLS) e poi, successivamente, dopo aver esportato l'IP, è stata prevista sintesi e implementazione in Xilinx\textsuperscript{\textregistered} Vivado\textsuperscript{\textregistered}. 
\\
Oltre a ciò, si specifica che per ogni implementazione, effettuata in Xilinx\textsuperscript{\textregistered} Vivado\textsuperscript{\textregistered}, è stato previsto un constraint di clock associato e corrispondente alla solution progettata in Xilinx\textsuperscript{\textregistered} Vivado\textsuperscript{\textregistered} High-Level Synthesis (HLS) e, in aggiunta, ogni design implementato è stato simulato considerando input random. Pertanto, generando un file \textit{.saif}, è stato possibile ottenere dati di potenza dinamica accurati.
